%% 
%% Copyright 2007-2020 Elsevier Ltd
%% 
%% This file is part of the 'Elsarticle Bundle'.
%% ---------------------------------------------
%% 
%% It may be distributed under the conditions of the LaTeX Project Public
%% License, either version 1.2 of this license or (at your option) any
%% later version.  The latest version of this license is in
%%    http://www.latex-project.org/lppl.txt
%% and version 1.2 or later is part of all distributions of LaTeX
%% version 1999/12/01 or later.
%% 
%% The list of all files belonging to the 'Elsarticle Bundle' is
%% given in the file `manifest.txt'.
%% 
%% Template article for Elsevier's document class `elsarticle'
%% with harvard style bibliographic references

%\documentclass[12pt]{elsarticle}

%% Use the option review to obtain double line spacing
%% \documentclass[preprint,review,12pt]{elsarticle}

%% Use the options 1p,twocolumn; 3p; 3p,twocolumn; 5p; or 5p,twocolumn
%% for a journal layout
%% \documentclass[final,1p,times]{elsarticle}
%% \documentclass[final,1p,times,twocolumn]{elsarticle}
%% \documentclass[final,3p,times]{elsarticle}
\documentclass[final,3p,times,twocolumn]{elsarticle}
%% \documentclass[final,5p,times]{elsarticle}
%% \documentclass[final,5p,times,twocolumn]{elsarticle}

%% For including figures, graphicx.sty has been loaded in
%% elsarticle.cls. If you prefer to use the old commands
%% please give \usepackage{epsfig}

%% The amssymb package provides various useful mathematical symbols
\usepackage{amssymb}
%% The amsthm package provides extended theorem environments
%% \usepackage{amsthm}

%% The lineno packages adds line numbers. Start line numbering with
%% \begin{linenumbers}, end it with \end{linenumbers}. Or switch it on
%% for the whole article with \linenumbers.
%% \usepackage{lineno}

\journal{Current Opinion in Solid State and Materials Science}

\begin{document}

\begin{frontmatter}

%% Title, authors and addresses

%% use the tnoteref command within \title for footnotes;
%% use the tnotetext command for theassociated footnote;
%% use the fnref command within \author or \address for footnotes;
%% use the fntext command for theassociated footnote;
%% use the corref command within \author for corresponding author footnotes;
%% use the cortext command for theassociated footnote;
%% use the ead command for the email address,
%% and the form \ead[url] for the home page:
%% \title{Title\tnoteref{label1}}
%% \tnotetext[label1]{}
%% \author{Name\corref{cor1}\fnref{label2}}
%% \ead{email address}
%% \ead[url]{home page}
%% \fntext[label2]{}
%% \cortext[cor1]{}
%% \affiliation{organization={},
%%             addressline={},
%%             city={},
%%             postcode={},
%%             state={},
%%             country={}}
%% \fntext[label3]{}

\title{Sampling thermodynamic ensembles of molecular systems with generative neural networks: Can we close the generalization gap?}

%% use optional labels to link authors explicitly to addresses:
\author[label1,label2]{Grant M. Rotskoff}
\affiliation[label1]{organization={Department of Chemistry, Stanford University},
            %addressline={Keck Science Building},
            city={Stanford},
            postcode={94305},
            state={CA},
            country={USA}}
%%
\affiliation[label2]{organization={Institute for Computational and Mathematical Engineering},
            %addressline={},
            city={Stanford},
            postcode={94305},
            state={CA},
            country={USA}}

% \affiliation{organization={Stanford University},%Department and Organization
%             addressline={}, 
%             city={},
%             postcode={}, 
%             state={},
%             country={}}

\begin{abstract}
%% Text of abstract

\end{abstract}

%%Graphical abstract
\begin{graphicalabstract}
%\includegraphics{grabs}
\end{graphicalabstract}

%%Research highlights
\begin{highlights}
\item Research highlight 1
\item Research highlight 2
\end{highlights}

\begin{keyword}
%% keywords here, in the form: keyword \sep keyword

%% PACS codes here, in the form: \PACS code \sep code

%% MSC codes here, in the form: \MSC code \sep code
%% or \MSC[2008] code \sep code (2000 is the default)

\end{keyword}

\end{frontmatter}

%% \linenumbers

%% main text
\section{Introduction and background}
\label{sec:intro}


Supervised machine learning has a rich history in the chemical sciences. 
For many decades, fitting appropriate regressive models to experimental data has been a central modality in the practice of rigorous, data-driven science.
In the last decade, the complexity and expressiveness of the available models has grown rapidly as the library of neural network architectures expanded to include models well-calibrated for chemical and dynamical data. 
Furthermore, one cannot discount the formative role played by increasingly powerful computational resources and sophisticated and reliable software. 
Together, these developments have led to 

A comparably new modeling paradigm is emerging alongside the developments in supervised learning due to the extraordinary developments in generative modeling
For the purposes of this review, a generative model is any model that seeks to learn a target probability distribution
Generative models have attracted enormous attention for applications outside of science, such as text to image generation with denoising diffusion models and autoregressive text generation with large language models. 
The natural question, of course, is whether or not these approaches can amplify and accelerate scientific inquiry.

In this review, we seek to articulate the potentialities and challenges of using 

- natural question is whether they can be used to accelerate molecular simulation

- this requires 

\section{Generative modeling for molecular systems}

\subsection{Models trained purely with data}

\subsection{Models trained with variational inference}


\section{Direct density estimation with normalizing flows}

- energy based training more natural, but suffers from mode collapse

- data based training more interpolative, so less to learn from the model

- Boltzmann generators were an unmitigated disaster that misled everyone and are still confusing the literature

\section{Augmenting with coarse-grained modeling}


\section{Alternative approaches; skipping replica exchange}



%% The Appendices part is started with the command \appendix;
%% appendix sections are then done as normal sections
%% \appendix

%% \section{}
%% \label{}

%% For citations use: 
%%       \citet{<label>} ==> Jones et al. [21]
%%       \citep{<label>} ==> [21]
%%

%% If you have bibdatabase file and want bibtex to generate the
%% bibitems, please use
%%
%%  \bibliographystyle{elsarticle-num-names} 
%%  \bibliography{<your bibdatabase>}

%% else use the following coding to input the bibitems directly in the
%% TeX file.

\begin{thebibliography}{00}

%% \bibitem[Author(year)]{label}
%% Text of bibliographic item

\bibitem[ ()]{}

\end{thebibliography}
\end{document}

\endinput
%%
%% End of file `elsarticle-template-num-names.tex'.
